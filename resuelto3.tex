% !TEX TS-program = pdflatex
% !TEX encoding = UTF-8 Unicode

% This is a simple template for a LaTeX document using the "article" class.
% See "book", "report", "letter" for other types of document.

% align equations to the left
% use larger type; default would be 10pt
\documentclass[fleqn, 11pt]{article}

\usepackage[utf8]{inputenc} % set input encoding (not needed with XeLaTeX)

%%% Examples of Article customizations
% These packages are optional, depending whether you want the features they
% provide.
% See the LaTeX Companion or other references for full information.

%%% PAGE DIMENSIONS
\usepackage{geometry} % to change the page dimensions
\geometry{a4paper} % or letterpaper (US) or a5paper or....
% for example, change the margins to 2 inches all round
% \geometry{margin=2in}
% set up the page for landscape
% \geometry{landscape}
%   read geometry.pdf for detailed page layout information

% support the \includegraphics command and options
\usepackage{graphicx}

% Activate to begin paragraphs with an empty line rather than an indent
\usepackage[parfill]{parskip}

%%% PACKAGES
% for much better looking tables
\usepackage{booktabs}
% for better arrays (eg matrices) in maths
\usepackage{array}
% very flexible & customisable lists (eg. enumerate/itemize, etc.)
\usepackage{paralist}
% adds environment for commenting out blocks of text & for better verbatim
\usepackage{verbatim}
% make it possible to include more than one captioned figure/table in a single
% float
\usepackage{subfig}
% These packages are all incorporated in the memoir class to one degree or
% another...

%%% HEADERS & FOOTERS
\usepackage{fancyhdr} % This should be set AFTER setting up the page geometry
\pagestyle{fancy} % options: empty , plain , fancy
\renewcommand{\headrulewidth}{0pt} % customise the layout...
\lhead{}\chead{}\rhead{}
\lfoot{}\cfoot{\thepage}\rfoot{}

%%% SECTION TITLE APPEARANCE
\usepackage{sectsty}
% (See the fntguide.pdf for font help)
\allsectionsfont{\sffamily\mdseries\upshape}
% (This matches ConTeXt defaults)

%%% ToC (table of contents) APPEARANCE
% Put the bibliography in the ToC
\usepackage[nottoc, notlof, notlot]{tocbibind}
% Alter the style of the Table of Contents
\usepackage[titles, subfigure]{tocloft}
\renewcommand{\cftsecfont}{\rmfamily\mdseries\upshape}
\renewcommand{\cftsecpagefont}{\rmfamily\mdseries\upshape} % No bold!

\usepackage{calc}
\usepackage{lmodern}
\usepackage{amssymb}
\usepackage{amsmath}
\usepackage{mathdots}

\newcommand{\nat}{\mathbb{N}}
\newcommand{\Ccur}{\mathcal{C}}
\newcommand{\Scur}{\mathcal{S}}
\newcommand{\indef}{\uparrow}
\newcommand{\ddef}{\downarrow}
\newcommand{\into}{\leftarrow}
\newcommand{\textcommand}[2]{\newcommand{#1}{\text{#2}}}
\textcommand{\IF}{IF}
\textcommand{\GOTO}{GOTO}
\textcommand{\WHILE}{WHILE}
\textcommand{\THEN}{THEN}
\textcommand{\ELSE}{ELSE}
\textcommand{\Dom}{Dom }
\textcommand{\Img}{Im }
\newcommand{\STP}[1][n]{\text{STP}^{(#1)}}
\newcommand{\SNAP}[1][n]{\text{SNAP}^{(#1)}}

\overfullrule=2em

%%% END Article customizations

%%% The "real" document content comes below...

\title{Resueltos Lógica y Computabilidad}
\author{Ignacio E. Losiggio}
% Activate to display a given date or no date (if empty), otherwise the current
% date is printed 
%\date{}

\begin{document}
\maketitle
\section{Práctica 3 --- Funciones no-computables y conjuntos c.e.}

\subsection{Probar, usando una diagonalización, que las siguientes funciones no
son computables:}

\begin{align*}
	f_1(x, y) &=
	\begin{cases}
		1 & \text{si } \Phi^{(1)}_x(y) \ddef \\
		0 & \text{en otro caso}
	\end{cases}
	&f_3(x, y, z) &=
	\begin{cases}
		1 & \text{si } \Phi^{(1)}_x(y) \ddef
		    \text{y } \Phi^{(1)}_x(y) > z \\
		0 & \text{en otro caso}
	\end{cases} \\
	f_2(x, y) &=
	\begin{cases}
		1 & \text{si } \Phi^{(1)}_x(y) = 0 \\
		0 & \text{en otro caso}
	\end{cases}
	&f_4(x) &=
	\begin{cases}
		1 & \text{si } \Phi^{(1)}_x(x) \ddef
		    \text{y } \Phi^{(1)}_x(x) \neq x \\
		0 & \text{en otro caso}
	\end{cases}
\end{align*}

\subsubsection{$f_1(x, y)$}

Si $f_1$ es computable entonces puedo construir el siguiente programa que lo
la use a cuyo número llamaremos $e$:
\[
	[A]\ \IF\ f_1(X_1, X_1) \neq 0\ \GOTO\ A
\]

Luego intentemos determinar el valor de $f_1(e, e)$:
\[
	f_1(e, e) = 1
	\iff \Phi^{(1)}_e(e) \ddef
	\iff f_1(e, e) = 0
\]

\subsubsection{$f_2(x, y)$}

Armemos de vuelta una función que sea molesta:
\begin{align*}
	[A]\ &\IF\ f_2(X_1, X_1) = 0\ \GOTO\ A \\
	     &Y \into Y + 1
\end{align*}

E intentemos determinar el valor de $f_2(e, e)$ otra vez:
\begin{gather*}
	f_2(e, e) = 1 \iff \Phi^{(1)}_e(e) = 0 \\
	f_2(e, e) = 0
	\iff \Phi^{(1)}_e(e) \neq 0 \lor \Phi^{(1)}_e(e) \indef
	\iff f_2(e, e) \neq 0 \lor f_2(e, e) \indef
\end{gather*}

\subsubsection{$f_3(x, y, z)$}

Cómo venimos haciendo empezamos por el ``programa rebelde'':
\begin{align*}
	[A]\ &\IF\ f_3(X_1, X_1, X_1) \neq 0\ \GOTO\ A \\
	     &Y \into X_1 + 1
\end{align*}

Y intentamos ver que determinar el valor de $f_3(e, e, e)$ no tiene sentido:
\begin{align*}
	f_3(e, e, e) = 1
	\iff& \Phi^{(1)}_e(e) \ddef \land\ \Phi^{(1)}_e(e) > e \\
	\iff& f_3(e, e, e) = 0\ \land\ f_3(e, e, e) = 0 \\
	\iff& f_3(e, e, e) = 0
\end{align*}

\subsubsection{$f_x(x)$}

\begin{align*}
	[A]\ &\IF\ f_4(X_1) \neq 0\ \GOTO\ A \\
	     &Y \into X_1 + 1
\end{align*}
\begin{align*}
	f_4(e) = 1
	\iff& \Phi^{(1)}_e(e) \ddef \land\ \Phi^{(1)}_e(e) \neq e \\
	\iff& f_4(e) = 0\ \land\ f_4(e) = 0 \iff f_4(e) = 0
\end{align*}

\subsection{Probar, reduciendo a cualquier función del ejercicio 1, que las
siguiente funciones no son computables:}

\begin{align*}
	g_1(x, y) =&
	\begin{cases}
		1 & \text{si } \Phi^{(1)}_x(y) \indef \\
		0 & \text{en otro caso}
	\end{cases} \\
	g_2(x, y, z, w) =&
	\begin{cases}
		1 & \text{si } \Phi^{(1)}_x(z) \ddef \text{y }
		    \Phi^{(1)}_y(w) \ddef \text{y }
		    \Phi^{(1)}_x(z) > \Phi^{(1)}_y(w) \\
		0 & \text{en otro caso}
	\end{cases} \\
	g_3(x, y, z) =&
	\begin{cases}
		z + 1 & \text{si } \Phi^{(1)}_x(y) \ddef \text{y }
		        \Phi^{(1)}_x(y) \neq z \\
		0     & \text{en otro caso}
	\end{cases} \\
	g_4(x, y, z) =&
	\begin{cases}
		(\Phi^{(1)}_x \circ \Phi^{(1)}_y)(z)
		  & \text{si } \Phi^{(1)}_y(z) \ddef \text{y }
		    (\Phi^{(1)}_x \circ \Phi^{(1)}_y)(z) \ddef \\
		0 & \text{en otro caso}
	\end{cases} \\
\end{align*}

\subsubsection{$g_1(x, y)$}

\[
	\alpha \circ g_1 = f_1
\]

\subsubsection{$g_2(x, y, z, w)$}

Sea $e$ el número de un programa que computa la función identidad podemos decir:
\[
	g_2(x, e, y, z) = f_2(x, y, z)
\]

\subsubsection{$g_3(x, y, z)$}

\[
	\beta(g_3(x, x, x)) = f_4(x)
\]

\subsubsection{$g_4(x, y, z)$}

Sea (otra vez) $e$ el número de un program que computa la función identidad
podemos decir:
\[
	g_4(x, e, y) = f_1(x, y)
\]

\subsection{Probar que la siguiente función no es computable reduciendo la
función $f_4$ del primer ejercicio}

\[
	g_3'(x, y, z) =
	\begin{cases}
		z & \text{si } \Phi^{(1)}_x(y) \ddef \text{y }
		    \Phi^{(1)}_x(y) \neq z \\
		0 & \text{en otro caso}
	\end{cases}
\]
\emph{Sugerencia}: Revisar que la reducción maneje correctamente el caso
$f_4(0)$.

No sé si esto está bien pero yo supondría que:
\[
	\beta(g_3'(x, x, x)) = f_4(x)
\] 
Dado que $0$ es el programia vacío (una implementación de $n(x)$) entonces que
$g_4(0, 0, 0) = 0$ se corresponde con el valor de $f_4(0)$. En el resto de los
valores de $x$ es evidente que ambas funciones hacen lo mismo.

\subsection{Decimos que una función parcial computable $f$ es \emph{extensible}
si existe $g$ computable tal que $g(x) = f(x)$ para todo $x \in \Dom f$. Probar
que existe una función parcial computable que no es extensible
(\emph{Sugerencia}: considerar una función tal que con su extensión se podría
computar alguna variante del halting problem).}

Vamos a tomar la siguiente $f$ que suponemos extensible:
\[
	f(x) =
	\begin{cases}
		\Phi^{(1)}_x(x) & \text{si } \Phi^{(1)}_x(x) \ddef \\
		\indef & \text{sino}
	\end{cases}
\]

Como es extensible entonces existe un $g$ que puede calcular resultados para
todo su dominio y devuelve ``basura'' para cualquier otro valor. Dado que $g$
es computable podemos definir el siguiente programa y tomar su número $e$:
\begin{gather*}
	Y \into X_1 \\
	\IF\ g(X_1) \neq X_1\ \GOTO\ E \\
	Y \into Y + 1
\end{gather*}

Cómo $g$ es total entonces nuestro programa con número $e$ también lo es (por
lo que $g(e) = f(e)$), entonces nos preguntamos: ¿Cuál es el valor de
$\Phi^{(1)}_e(e)$? Sabemos que puede ser o bien $e$ o $e + 1$.
\begin{align*}
	\Phi^{(1)}_e(e) = e \iff g(e) \neq e \iff \Phi^{(1)}_e(e) \neq e \\
	\Phi^{(1)}_e(e) = e + 1 \iff g(e) = e \iff \Phi^{(1)}_e(e) = e
\end{align*}



\end{document}
