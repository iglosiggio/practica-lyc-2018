% !TEX TS-program = pdflatex
% !TEX encoding = UTF-8 Unicode

% This is a simple template for a LaTeX document using the "article" class.
% See "book", "report", "letter" for other types of document.

\documentclass[11pt]{article} % use larger type; default would be 10pt

\usepackage[utf8]{inputenc} % set input encoding (not needed with XeLaTeX)

%%% Examples of Article customizations
% These packages are optional, depending whether you want the features they provide.
% See the LaTeX Companion or other references for full information.

%%% PAGE DIMENSIONS
\usepackage{geometry} % to change the page dimensions
\geometry{a4paper} % or letterpaper (US) or a5paper or....
% \geometry{margin=2in} % for example, change the margins to 2 inches all round
% \geometry{landscape} % set up the page for landscape
%   read geometry.pdf for detailed page layout information

\usepackage{graphicx} % support the \includegraphics command and options

% \usepackage[parfill]{parskip} % Activate to begin paragraphs with an empty line rather than an indent

%%% PACKAGES
\usepackage{booktabs} % for much better looking tables
\usepackage{array} % for better arrays (eg matrices) in maths
\usepackage{paralist} % very flexible & customisable lists (eg. enumerate/itemize, etc.)
\usepackage{verbatim} % adds environment for commenting out blocks of text & for better verbatim
\usepackage{subfig} % make it possible to include more than one captioned figure/table in a single float
% These packages are all incorporated in the memoir class to one degree or another...

%%% HEADERS & FOOTERS
\usepackage{fancyhdr} % This should be set AFTER setting up the page geometry
\pagestyle{fancy} % options: empty , plain , fancy
\renewcommand{\headrulewidth}{0pt} % customise the layout...
\lhead{}\chead{}\rhead{}
\lfoot{}\cfoot{\thepage}\rfoot{}

%%% SECTION TITLE APPEARANCE
\usepackage{sectsty}
\allsectionsfont{\sffamily\mdseries\upshape} % (See the fntguide.pdf for font help)
% (This matches ConTeXt defaults)

%%% ToC (table of contents) APPEARANCE
\usepackage[nottoc,notlof,notlot]{tocbibind} % Put the bibliography in the ToC
\usepackage[titles,subfigure]{tocloft} % Alter the style of the Table of Contents
\renewcommand{\cftsecfont}{\rmfamily\mdseries\upshape}
\renewcommand{\cftsecpagefont}{\rmfamily\mdseries\upshape} % No bold!

\usepackage{amssymb}
\usepackage{amsmath}
\usepackage{mathdots}
\setlength\parindent{0pt}
\setlength{\parskip}{3mm}

\newcommand{\desarrollo}[1]{

    \hspace{2em}
    \begin{minipage}{\textwidth}
        #1
    \end{minipage}

}
\newcommand{\nat}{\mathbb{N}}
\newcommand{\Ccur}{\mathcal{C}}

%%% END Article customizations

%%% The "real" document content comes below...

\title{Resueltos Lógica y Computabilidad}
\author{Ignacio E. Losiggio}
%\date{} % Activate to display a given date or no date (if empty),
         % otherwise the current date is printed 

\begin{document}
\maketitle

\section{Práctica 1 - Funciones primitivas recursivas y clases PRC}

\subsection{Mostrar que, dado un $k$ fijo, la función constante $f(x)=k$ puede definirse usando las funciones iniciales y composición (sin usar recursión primitiva)}

Podemos empezar bien qué funciones nos piden, si $k = 0$  entonces $f_0(x)=n(x)$ (con $n$ la función constante $0$, que es primitiva). Si $k = 1$ entonces $f_1 = s(f_0(x))$, que sabemos que es primitiva recursiva porque lo hicimos una composición válida. Podemos intentar plasmar entonces estas funciones que nos piden en búqueda de un patrón:
\desarrollo{
\begin{tabular}{lll}
$k$     & $f_k(x)$                 & Expresión expandida \\ \hline
$k = 0$ & $f_0(x) = n(x)$          & $n(x)$ \\
$k = 1$ & $f_1(x) = s(f_0(x))$     & $s(n(x))$ \\
$k = 2$ & $f_2(x) = s(f_1(x))$     & $s(s(n(x))$ \\
$k = n$ & $f_n(x) = s(f_{n-1}(x))$ & $s(s(...s(n(x))))$ \\
\end{tabular}
}
Cómo podemos ver, construir la función constantemente $k$ sólo requiere de aplicar $s(x)$ $k$veces sobre $n(x)$.

\subsection{Probar que las siguientes funciones son primitivas recursivas, mostrando que pueden obtenerse a partir de las funciones iniciales usando composición t/o recursión primitiva:}

\subsubsection{$f_1(x,y)=x+y$}

Tomamos la estructura de la recursión primitiva:
\desarrollo{$
f(x_1,...,x_n,0)   = h(x_1,...,x_n) \\
f(x_1,...,x_n,t+1) = g(f(x_1,...,x_n,t),x_1,...,x_n,t)
$}

Con esto tenemos que elegir una $h(x)$ y una $g(n,x,t)$ que nos construyan la suma y copypastear cómo corresponda:
\desarrollo{$
f_1(x,0)   = u^1_1(x) \\
f_1(x,t+1) = g(f_1(x,t),x,t) \\
g(n,x,t)   = s(u^3_1(n,x,t))
$}

¿No queda claro por qué funciona? Hagamos un par de reemplazos a ver que pasa.
\desarrollo{$
g(n,x,t)   = s(u^3_1(n,x,t)) = s(n) \\
f_1(x,0)   = u^1_1(x) = x \\
f_1(x,t+1) = g(f_1(x,t),x,t) = s(f_1(x,t))
$}

Ajá!
\desarrollo{
\begin{tabular}{m{5cm}|l}
Nuestros reemplazos & Una forma de describir la suma \\ \hline \\[-7pt]
$
f_1(x,0)   = x \newline
f_1(x,t+1) = s(f_1(x,t))
$ & $
f_1(x,y) = 
\begin{cases}
  x             & \text{si } y = 0 \\
  1 + f_1(x, y-1) & \text{sino}
\end{cases}
$
\end{tabular}
}

Bueno, con este truco vamos a meterle a todo el resto de los ejercicios de este punto!

\subsubsection{$f_2(x,y)=x \cdot y$}

Bien, pongamos de vuelta en práctica lo que acabamos de hacer. Si tu intuición te llama a que vamos a tener que usea $f_1(x,y)$ acá estás más que en lo cierto. Pero primero plantiemos una función partida común y corriente a ver cómo vamos desde ahí a nuestra recursión primitiva:
\desarrollo{$
f_2(x,y) = 
\begin{cases}
  0             & \text{si } y = 0 \\
  x + f_2(x, y-1) & \text{sino}
\end{cases}
$}

Y de ahí salimos a buscar nuestro $h(x)$ y nuestro $g(n,x,t)$:
\desarrollo{$
f_2(x,0)   = n(x) = 0 \\
f_2(x,t+1) = g(f_2(x,t),x,t) \\
g(n,x,t)   = f_1(u^3_2(n,x,t),u^3_1(n,x,t)) = x + n
$}

Y para estar del todo seguros hacemos todos los reemplazos cómo la otra vez:
\desarrollo{$
f_2(x,0)   = 0\\
f_2(x,t+1) = x + f_2(x,t)
$}

Justo lo que buscábamos.

\subsubsection{$f_3(x,y)=x^y$}

Bueno, supongo que ya te diste cuenta de cuál es el patrón, te paso cuál es la $h(x)$ y la $g(n,x,t)$:
\desarrollo{$
h(x)     = s(n(x)) = 1 \\
g(n,x,t) = f_2(u^3_2(n,x,t), u^3_1(n,x,t) = x \cdot n \\
\\
f_3(x,0)   = 1 \\
f_3(x,t+1) = x \cdot f_3(x,t)
$}

\subsubsection{$f_4(x,y)=x^{x^{\cdot^{\cdot^{\cdot^{^{x}}}}}}$}

Observación: se asume que $f_4(x,0)=1$
\desarrollo{$
h(x) = s(n(x)) = 1 \\
g(n,x,t) = f_3(u^3_2(n,x,t),u^3_1(n,x,t)) = x^n \\
\\
f_4(x,0) = 1 \\
f_4(x,t+1) = x^{f_4(x,t)}
$}

\subsubsection{$g_1(x)=\dot-1$}

Cómo sólo estamos operando en los $\nat$ el predecesor de 0 es 0. Una cosa a notar es que (según la teórica, filmina 25, primera clase de computabilidad) ``En este contexto, una función 0-aria es una constante $k$. Si $f$ es 1-aria y $t=0$ entonces $h(t)=k=s^{(k)}(n(t))$.".
\desarrollo{$
g_1(0) = n(0)\\
g_1(t+1) = u^1_1(t)
$}

\subsubsection{$g_2(x,y)=x\dot-y$}

Observación:
\desarrollo{$
x\dot-y =
\begin{cases}
  x - y & \text{si } y \le x \\
  0     & \text{si } y > x
\end{cases}
$}

Bueno, entonces podemos empezar a intentar construir la resta.
\desarrollo{$
f(x) = u^1_1(x) = x \\
g(n, x, t) = g_1(u^3_1(n,x,t)) = n \dot- 1
$}

¿Está bien esto? Probemos algún número:
\desarrollo{$
g_2(5, 3) = g(g_2(5, 2), 5, 3) = g_2(5, 2) - 1 \\
g_2(5, 3) = g(g_2(5, 1), 5, 3) - 1 = g_2(5, 1) - 1 - 1 \\
g_2(5, 3) = g(g_2(5, 0), 5, 3) - 1 - 1 = g_2(5, 0) - 1 - 1 - 1 \\
g_2(5, 3) = 5 - 1 - 1 - 1 \\
g_2(5, 3) = 2
$}

¡Bien!

\subsubsection{$g_3(x,y)=max\{x,y\}$}

Cómo operamos con los naturales sabemos que si $x < y$ entonces $x - y = 0$ podemos usar la recursión primitiva para construir una función de decisión.
\desarrollo{$
d(x, y, 0) = u^2_2(x, y) = y \\
d(x, y, t+1) = u^4_2(d(x, y, t), x, y, t) = x
$}

Ahora, esto no es $g_3(x,y)$ pero está peligrosamente cerca, podemos usar la composición para finalmente construirlo.
\desarrollo{$
g_3(x,y) = d(u^2_1(x, y), u^2_2(x, y), g_2(x,y)) = d(x, y, x \dot- y)
$}

\subsubsection{$g_4(x,y)=min\{x,y\}$}

La idea es la misma que recién, pero construímos parametrizamos la decisión al revés:
\desarrollo{$
g_3(x,y) = d(u^2_2(x,y), u^2_1(x,y), g_2(x,y)) = d(y, x, x \dot- y)
$}

\subsection{Sea $\Ccur_i$ la clase de funciones iniciales y $\Ccur_c$ la (mínima) clase que extiende a $\Ccur_i$ y se encuentra cerrada por composición:}

Es decir, $\Ccur_i$ aquella que contiene a:

\begin{tabular}{lcr}
$n(x)=0$, &
$s(x)=x+1$, &
$u^n_i(x_1,...x_n)=x_i$ para cada $n \in \nat$ e $i \in \{1, ..., n\}$
\end{tabular}

Y $\Ccur_c$ aquella que si $f, g_1, ..., g_m$ están en $\Ccur_c$, entonces $h(x_1, ..., x_n) = f(g_1(x_1, ..., x_n), ..., g_m(x_1, ..., x_n))$ también lo está.

\subsubsection{Demostrar que para toda $f: \nat^n \to \nat$, $f$, está en $\Ccur_c$ sii existe $k \geq 0$ tal que, o bien sucede $f(x_1, ..., x_n) = k$, o bien para algún $i$ fijo, se tiene $f(x_1, ..., x_n) = x_i + k$.}

El enunciado nos propone demostrar que $\Ccur_c$ sólo tiene funciones que devuelven constantes y funciones que suman constantes a \emph{uno} de sus parámetros (y siempre el mismo).

Demostrar la vuelta es particularmente fácil, primero construyo la función 1-aria que suma $k$ con el método del primer ejercicio (teniendo $u^1_1(x)$ cómo caso base en lugar de $n(x)$ y la llamo $k(x)$. También construyo una función constantemente 0 que tome $n$ argumentos.
\desarrollo{$
k(x) = s(...s(u^1_1(x))) \text{\hspace{7em} Nota: si $k=0$ entonces } k(x)=u^1_1(x) \\
z(x_1, ...., x_n) = n(u^n_1(x_1, ..., x_n))
$}

Ahora sólo tengo que cubrir cada caso:
\begin{itemize}
\item Si tengo una $f: \nat^n \to \nat$ que es constantemente $k$.
\desarrollo{$ f(x_1, ..., x_n) = k(z(x_1, ..., x_n)) $}
\item Si tengo una $f: \nat^n \to \nat$ que es constantemente $x_i + k$.
\desarrollo{$ f(x_1, ..., x_n) = k(u^n_i(x_1,...,x_n)) $}
\end{itemize}

Cómo sólo hice composiciones para construir la $f$ entonces esto debería bastar para demostrar que las funciones constantes y las que suman constantes pertenecen todas a $\Ccur_c$.

Para demostrar la ida necesito asegurarme que las funciones en $\Ccur_c$ sólo pueden ser constantes (o sumar constantes). Si empezamos examinando los casos base...
\desarrollo{
\begin{tabular}{lll}
Primitiva                  & Forma                        & $k$ \\ \hline \\[-1em]
$n(x)=0$                   & $f(x) = k$                   & $0$ \\
$s(x)=x+1$                 & $f(x) = x + k = x + 1$       & $1$ \\
$u^n_i(x_1, ..., x_n)=x_i$ & $f(x_1, ..., x_n) = x_i + k$ & $0$
\end{tabular}
}

...tenemos que todos ellos tienen o la forma $k$ o la de $x_i + k$. Ahora examinemos nuestra regla de composición:
\desarrollo{
Sea:
\desarrollo{$
f: \nat^m \to \nat \\
g_1, ..., g_m: \nat^n \to \nat
$}
Podemos construir $h: \nat^n \to \nat$ de la siguiente manera:
\desarrollo{$h(x_1, ..., x_n) = f(g_1(x_1, ..., x_n), ..., g_m(x_1, ..., g_n))$}
}

Puedo tomar $f_1, ..., f_n$ que tengan la forma $k$ o la forma $x_i + k$  y analizar qué pasa si las compongo (paso inductivo).
\desarrollo{
\begin{tabular}{l|l|c|l|c}
& \multicolumn{2}{c|}{$f_i(x_1, ..., x_n) = k$} & \multicolumn{2}{c}{$f_i(x_1, ..., x_n) = x_j + k$} \\ \cline{2-5}
&&&&\\[-1em] 
Composición & Forma &$k'$ & Forma & $k'$ \\ \hline
&&&&\\[-1em]
$n(f_i(x_1,...,f_n))$ & $f(x_1,...,x_n)=k$ & $0$ & $f(x_1,...,x_n)=k$ & $0$ \\
$s(f_i(x_1,...,f_n))$ & $f(x_1,...,x_n)=k$ & $k+1$ & $f(x_1,...,x_n)=x_j+k$ & $k+1$ \\
$u^n_i(f_1(...),...,f_i(...),...,f_n(...))$ & $f(x_1,...,x_n)=k$ & $k$ & $f(x_1,...,x_n)=x_j+k$ & k \\
$f_i(f_1(...),...,f_n(...))$ & $f(x_1,...,x_n)=k$ & $k$ & $f(x_1,...,x_n)=f_j(...)+k$ & $f_j(...)+k$
\end{tabular}
}

Si bien el último caso puede parecer molesto, por la hipótesis inductiva sabemos que $f_k$ tiene o bien forma $k$ o bien forma $x_i+k$, entonces $k'=k_{f_j}+k_{f_i}$ y la forma se mantiene.

Entonces, si parto de funciones en $\Ccur_c$ y las compongo siempre voy a llegar a una función de la forma $k$ o $x_i+k$. Y dado que las funciones iniciales tienen también esa forma entonces todo $\Ccur_c$ la tiene también.

\subsubsection{Mostrar que existe una función primitiva recursiva que no está en $\Ccur_c$}

$f_1(x,y)$ del ejercicio 2. Es primitiva recursiva porque está construida en base a la recursión primitiva y a la composición de funciones. Pero depende de ambos parámetros para emitir su resultado (a diferencia de todas las de $\Ccur_c$).

\subsection{Mostrar que los predicados $\leq$, $\geq$, $=$, $\neq$, $<$ t $>$ $: \nat^2\to\{0,1\}$ están en cualquier clase \emph{PRC}}

Llamamos \emph{predicado} a cualquier función $p: \nat^n\to\{0,1\}$, escribimos $p(a_1,...,a_n)$ en lugar de $p(a_1,...,a_n)=1$ y decimos, informalmente, en este caso, que ``$p(a_1,...,a_n)$ es verdadero".

Vamos a empezar poniéndoles nombres a las funciones que queremos construir y vamos a tener en cuenta las funciones que construímos en el ejercicio 2 para facilitarnos la vida:

\begin{tabular}{lcr}
$f_1(x,y)=
  \begin{cases}
  1 & \text{ si } x\leq y \\
  0 & sino
  \end{cases}$ &
$f_2(x,y)=
  \begin{cases}
  1 & \text{ si } x\geq y \\
  0 & sino
  \end{cases}$ &
$f_3(x,y)=
  \begin{cases}
  1 & \text{ si } x=y \\
  0 & sino
  \end{cases}$ \\ \\
$f_4(x,y)=
  \begin{cases}
  1 & \text{ si } x\neq y \\
  0 & sino
  \end{cases}$ &
$f_5(x,y)=
  \begin{cases}
  1 & \text{ si } x<y \\
  0 & sino
  \end{cases}$ &
$f_6(x,y)=
  \begin{cases}
  1 & \text{ si } x>y \\
  0 & sino
  \end{cases}$ \\
\end{tabular}

Y para hacernos la vida más fácil vamos a construir $\alpha(x)$ que es la negación lógica en nuestro modelo.
\desarrollo{$
\alpha(0) = n(0) \\
\alpha(t+1) = s(n(t))
$}

¡Y ahora sí!
\desarrollo{$
f_1(x,y)=\alpha(x\dot-y) \\
f_2(x,y)=\alpha(y\dot-x) \\
f_3(x,y)=(x \leq y) \cdot (y \leq x) \\
f_4(x,y)=\alpha(x = y) \\
f_5(x,y)=\alpha(x \geq y)) \\
f_6(x,y)=\alpha(x \leq y))
$}

\emph{Nota:} $f_3$ puede parecer raro al no parecer una composición tan simple, plantearlo prolijamente requiere una funcion auxiliar que cambia el orden de los parámetros.

\subsection{Sean $\Ccur$ una clase \emph{PRC}, $f_1,...,f_k,g:\nat^n\to\nat$ funciones en $\Ccur$ y $p_1,...,p_k:\nat^n\to\nat$ predicados disjuntos en $\Ccur$. Mostrar que la siguiente función también está en $\Ccur$:}

\desarrollo{$
h(x_1,...,x_n) =
\begin{cases}
  f_1(x_1,...,x_n) & \text{ si } p_1(x_1,...,x_n) \\
  \hspace{3em}\vdots \\
  f_k(x_1,...,x_n) & \text{ si } p_k(x_1,...,x_n) \\
  g(x_1,...,x_n) & \text{ sino} \\
\end{cases}
$}

Observar que $h$ queda completamente determinada por este esquema.

\emph{Nota:} Al ser $p_1,...,p_k$ disjuntos no sucede $p_i(a_1,...,a_n) = p_j(a_1,...,a_n) = 1$ con $i \neq j$ para ningún $(a_1,...,a_n) \in \nat^n$.

Si podemos resolverlo para el caso de $k=1$ entonces podremos resolverlo para cualquier $k$ arbitrario. Podemos construir $h(x_1,...,x_n)$ de la siguiente forma:
\desarrollo{$
h(x_1,...,x_n) = h_1(x_1,...,x_n) \\
h_i(x_1,...,x_n) =
\begin{cases}
  f_i(x_1,...,x_n) & \text{ si } p_i(x_1,...,x_n) \\
  h_{i+1}(x_1,...,x_n) & \text{ sino}
\end{cases}
\hspace{1em} \forall i \neq k+1 \\
h_{k+1}(x_1,...,x_n) = g(x_1,...,x_n)
$}

Ahora sólo queda resolverlo para el caso de 2. ¡Cosa que ya hicimos en el ejercicio 2 con $max\{x,y\}$ y $min\{x,y\}$! Repasemos esa solución: construimos una función de decisión $d_i$ y la encapsulamos para construir el $h_i$.
\desarrollo{$
d_i(x_1,...,x_n,0) = f_i(x_1,...,x_n) \\
d_i(x_1,...,x_n,t+1) = h_{i+1}(x_1,...,x_n) \\ \\
h_i(x_1,...,x_n) = d_i(x_1,...,x_n,p(x_1,...,x_n)) \hspace{1em} \forall i \neq k+1 \\
h_{k+1}(x_1,...,x_n) = g(x_1,...,x_n)
$}

Ahora tenemos todo listo, sólo queda enunciar a nuestro $h$:
\desarrollo{$h(x_1,...,x_n) = h_1(x_1,...,x_n)$}

\end{document}