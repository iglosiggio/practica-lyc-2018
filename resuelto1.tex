% !TEX TS-program = pdflatex
% !TEX encoding = UTF-8 Unicode

% This is a simple template for a LaTeX document using the "article" class.
% See "book", "report", "letter" for other types of document.

% align equations to the left
% use larger type; default would be 10pt
\documentclass[fleqn, 11pt]{article}

\usepackage[utf8]{inputenc} % set input encoding (not needed with XeLaTeX)

%%% Examples of Article customizations
% These packages are optional, depending whether you want the features they
% provide.
% See the LaTeX Companion or other references for full information.

%%% PAGE DIMENSIONS
\usepackage{geometry} % to change the page dimensions
\geometry{a4paper} % or letterpaper (US) or a5paper or....
% for example, change the margins to 2 inches all round
% \geometry{margin=2in}
% set up the page for landscape
% \geometry{landscape}
%   read geometry.pdf for detailed page layout information

% support the \includegraphics command and options
\usepackage{graphicx}

% Activate to begin paragraphs with an empty line rather than an indent
\usepackage[parfill]{parskip}

%%% PACKAGES
% for much better looking tables
\usepackage{booktabs}
% for better arrays (eg matrices) in maths
\usepackage{array}
% very flexible & customisable lists (eg. enumerate/itemize, etc.)
\usepackage{paralist}
% adds environment for commenting out blocks of text & for better verbatim
\usepackage{verbatim}
% make it possible to include more than one captioned figure/table in a single
% float
\usepackage{subfig}
% These packages are all incorporated in the memoir class to one degree or
% another...

%%% HEADERS & FOOTERS
\usepackage{fancyhdr} % This should be set AFTER setting up the page geometry
\pagestyle{fancy} % options: empty , plain , fancy
\renewcommand{\headrulewidth}{0pt} % customise the layout...
\lhead{}\chead{}\rhead{}
\lfoot{}\cfoot{\thepage}\rfoot{}

%%% SECTION TITLE APPEARANCE
\usepackage{sectsty}
% (See the fntguide.pdf for font help)
\allsectionsfont{\sffamily\mdseries\upshape}
% (This matches ConTeXt defaults)

%%% ToC (table of contents) APPEARANCE
% Put the bibliography in the ToC
\usepackage[nottoc, notlof, notlot]{tocbibind}
% Alter the style of the Table of Contents
\usepackage[titles, subfigure]{tocloft}
\renewcommand{\cftsecfont}{\rmfamily\mdseries\upshape}
\renewcommand{\cftsecpagefont}{\rmfamily\mdseries\upshape} % No bold!

\usepackage{calc}
\usepackage{lmodern}
\usepackage{amssymb}
\usepackage{amsmath}
\usepackage{mathdots}
\usepackage{mathtools}

\newcommand{\nat}{\mathbb{N}}
\newcommand{\Ccur}{\mathcal{C}}
\newcommand{\indef}{\downarrow}

\overfullrule=2em

%%% END Article customizations

%%% The "real" document content comes below...

\title{Resueltos Lógica y Computabilidad}
\author{Ignacio E. Losiggio}
%\date{} % Activate to display a given date or no date (if empty),
% otherwise the current date is printed 

\begin{document}
\maketitle

\section{Práctica 1 --- Funciones primitivas recursivas y clases PRC}

\subsection{Mostrar que, dado un $k$ fijo, la función constante $f(x) = k$
puede definirse usando las funciones iniciales y composición (sin usar
recursión primitiva)}

Podemos empezar bien qué funciones nos piden, si $k = 0$  entonces
$f_0(x) = n(x)$ (con $n$ la función constante $0$, que es primitiva). Si
$k = 1$ entonces $f_1 = s(f_0(x))$, que sabemos que es primitiva recursiva
porque lo hicimos una composición válida. Podemos intentar entonces
búqueda de un patrón dentro de las funciones que nos piden:
\begin{center}
	\begin{tabular}{lll}
		$k$     & $f_k(x)$ & Expresión expandida \\ \toprule
		$k = 0$ & $f_0(x) = n(x)$          & $n(x)$ \\
		$k = 1$ & $f_1(x) = s(f_0(x))$     & $s(n(x))$ \\
		$k = 2$ & $f_2(x) = s(f_1(x))$     & $s(s(n(x))$ \\
		$k = n$ & $f_n(x) = s(f_{n-1}(x))$ & $s(s(\dots s(n(x))))$ \\
	\end{tabular}
\end{center}

Cómo podemos ver, construir la función constantemente $k$ sólo requiere de
aplicar $s(x)$ $k$veces sobre $n(x)$.

\subsection{Probar que las siguientes funciones son primitivas recursivas,
mostrando que pueden obtenerse a partir de las funciones iniciales usando
composición t/o recursión primitiva:}

\subsubsection{$f_1(x, y) = x + y$}

Tomamos la estructura de la recursión primitiva:
\begin{align*}
	f(x_1, \dots, x_n, 0)     &= h(x_1, \dots, x_n) \\
	f(x_1, \dots, x_n, t + 1) &= g(f(x_1, \dots, x_n, t), x_1, \dots, x_n,
	t)
\end{align*}

Con esto tenemos que elegir una $h(x)$ y una $g(n, x, t)$ que nos construyan la
suma y copypastear cómo corresponda:
\begin{align*}
	f_1(x, 0)     &= u^1_1(x) \\
	f_1(x, t + 1) &= g(f_1(x, t), x, t) \\
	g(n, x, t)    &= s(u^3_1(n, x, t))
\end{align*}

¿No queda claro por qué funciona? Hagamos un par de reemplazos a ver que pasa.
\begin{align*}
	g(n, x, t)    &= s(u^3_1(n, x, t)) \\
	&= s(n) \\
	f_1(x, 0)     &= u^1_1(x) \\
	&= x \\
	f_1(x, t + 1) &= g(f_1(x, t), x, t) \\
	&= s(f_1(x, t))
\end{align*}

Ajá!
\begin{gather*}
	\begin{aligned}
		f_1(x, 0)     &= x \\
		f_1(x, t + 1) &= s(f_1(x, t))
	\end{aligned}
	\text{\ es equivalente a }
	f_1(x, y) = 
	\begin{cases}
		x                 & \text{si } y = 0 \\
		1 + f_1(x, y - 1) & \text{sino}
	\end{cases}
\end{gather*}

Bueno, con este truco vamos a meterle a todo el resto de los ejercicios de este
punto!

\subsubsection{$f_2(x, y) = x \cdot y$}

Bien, pongamos de vuelta en práctica lo que acabamos de hacer. Si tu intuición
te llama a que vamos a tener que usea $f_1(x, y)$ acá estás más que en lo
cierto. Pero primero plantiemos una función partida común y corriente a ver
cómo vamos desde ahí a nuestra recursión primitiva:
\begin{equation*}
	f_2(x, y) = 
	\begin{cases}
		0                 & \text{si } y = 0 \\
		x + f_2(x, y - 1) & \text{sino}
	\end{cases}
\end{equation*}

Y de ahí salimos a buscar nuestro $h(x)$ y nuestro $g(n, x, t)$:
\begin{align*}
	f_2(x, 0)     &= n(x) \\
	&= 0 \\
	f_2(x, t + 1) &= g(f_2(x, t), x, t) \\
	g(n, x, t)     &= f_1(u^3_2(n, x, t), u^3_1(n, x, t)) \\
	&= x + n
\end{align*}

Y para estar del todo seguros hacemos todos los reemplazos cómo la otra vez:
\begin{align*}
	f_2(x, 0)     &= 0 \\
	f_2(x, t + 1) &= x + f_2(x, t)
\end{align*}

Justo lo que buscábamos.

\subsubsection{$f_3(x, y) = x^y$}

Bueno, supongo que ya te diste cuenta de cuál es el patrón, te paso cuál es la
$h(x)$ y la $g(n, x, t)$:
\begin{align*}
	h(x)          &= s(n(x)) \\
	&= 1 \\
	g(n, x, t)    &= f_2(u^3_2(n, x, t), u^3_1(n, x, t)) \\
	&= x \cdot n \\
	\\
	f_3(x, 0)     &= 1 \\
	f_3(x, t + 1) &= x \cdot f_3(x, t)
\end{align*}

\subsubsection{$f_4(x, y) = x^{x^{\cdot^{\cdot^{\cdot^{^{x}}}}}}$}

Observación: se asume que $f_4(x, 0) = 1$
\begin{align*}
	h(x)       &= s(n(x)) \\
	&= 1 \\
	g(n, x, t) &= f_3(u^3_2(n, x, t), u^3_1(n, x, t)) \\
	&= x^n \\
	\\
	f_4(x, 0)     &= 1 \\
	f_4(x, t + 1) &= x^{f_4(x, t)}
\end{align*}

\subsubsection{$g_1(x) = x \dot- 1$}

Cómo sólo estamos operando en los $\nat$ el predecesor de 0 es 0. Una cosa a
notar es que (según la teórica, filmina 25, primera clase de computabilidad)
``En este contexto, una función 0-aria es una constante $k$. Si $f$ es 1-aria y
$t = 0$ entonces $h(t) = k = s^{(k)}(n(t))$.''.
\begin{align*}
	g_1(0)     &= n(0)\\
	g_1(t + 1) &= u^1_1(t)
\end{align*}

\subsubsection{$g_2(x, y) = x \dot- y$}

Observación:
\begin{align*}
	x \dot- y =
	\begin{cases}
		x - y & \text{si } y \le x \\
		0     & \text{si } y > x
	\end{cases}
\end{align*}

Bueno, entonces podemos empezar a intentar construir la resta.
\begin{align*}
	f(x)       &= u^1_1(x) \\
	&= x \\
	g(n, x, t) &= g_1(u^3_1(n, x, t)) \\
	&= n \dot- 1
\end{align*}

¿Está bien esto? Probemos algún número:
\begin{alignat*}{3}
	g_2(5, 3) &= g(g_2(5, 2), 5, 3)         &&= g_2(5, 2) - 1 \\
	g_2(5, 3) &= g(g_2(5, 1), 5, 3) - 1     &&= g_2(5, 1) - 1 - 1 \\
	g_2(5, 3) &= g(g_2(5, 0), 5, 3) - 1 - 1 &&= g_2(5, 0) - 1 - 1 - 1 \\
	g_2(5, 3) &= 5 - 1 - 1 - 1 \\
	g_2(5, 3) &= 2
\end{alignat*}

¡Bien!

\subsubsection{$g_3(x, y) = \max\{x, y\}$}

Como operamos con los naturales sabemos que si $x < y$ entonces $x - y = 0$
podemos usar la recursión primitiva para construir una función de decisión.
\begin{align*}
	d(x, y, 0)     &= u^2_2(x, y) \\
	&= y \\
	d(x, y, t + 1) &= u^4_2(d(x, y, t), x, y, t) \\
	&= x
\end{align*}

Ahora, esto no es $g_3(x, y)$ pero está peligrosamente cerca, podemos usar la
composición para finalmente construirlo.
\[
	\begin{split}
		g_3(x, y) &= d(u^2_1(x, y), u^2_2(x, y), g_2(x, y)) \\
		&= d(x, y, x \dot- y)
	\end{split}
\]

\subsubsection{$g_4(x, y) = \min\{x, y\}$}

La idea es la misma que recién, pero parametrizamos la decisión al revés:
\[
	\begin{split}
		g_3(x, y) &= d(u^2_2(x, y), u^2_1(x, y), g_2(x, y)) \\
		&= d(y, x, x \dot- y)
	\end{split}
\]

\subsection{Sea $\Ccur_i$ la clase de funciones iniciales y $\Ccur_c$ la
(mínima) clase que extiende a $\Ccur_i$ y se encuentra cerrada por
composición:}

Es decir, $\Ccur_i$ aquella que contiene a:

\begin{align*}
	n(x) = 0
	&&s(x) = x + 1
	&&u^n_i(x_1, \dots x_n)=x_i
	& \text{\ para cada } n \in \nat \text{\ e } i \in \{1, \dots, n\}
\end{align*}

Y $\Ccur_c$ aquella que si $f, g_1, \dots, g_m$ están en $\Ccur_c$, entonces
$h(x_1, \dots, x_n) =\\ f(g_1(x_1, \dots, x_n), \dots, g_m(x_1, \dots, x_n))$
también lo está.

\subsubsection{Demostrar que para toda $f: \nat^n \to \nat$, $f$, está en
$\Ccur_c$ sii existe $k \geq 0$ tal que, o bien sucede $f(x_1, \dots, x_n) =
k$, o bien para algún $i$ fijo, se tiene $f(x_1, \dots, x_n) = x_i + k$.}

El enunciado nos propone demostrar que $\Ccur_c$ sólo tiene funciones que
devuelven constantes y funciones que suman constantes a \emph{uno} de sus
parámetros (y siempre el mismo).

Demostrar la vuelta es particularmente fácil, primero construyo la función
1-aria que suma $k$ con el método del primer ejercicio (teniendo $u^1_1(x)$
cómo caso base en lugar de $n(x)$ y la llamo $k(x)$. También construyo una
función constantemente 0 que tome $n$ argumentos.
\begin{align*}
	k(x)                &= s(\dots s(u^1_1(x)))
	&\text{Nota: si }k = 0 \text{\ entonces } k(x) = u^1_1(x) \\
	z(x_1, \dots., x_n) &= n(u^n_1(x_1, \dots, x_n))
\end{align*}

Ahora sólo tengo que cubrir cada caso:
\begin{itemize}
	\item Si tengo una $f: \nat^n \to \nat$ que es constantemente $k$.
		\[f(x_1, \dots, x_n) = k(z(x_1, \dots, x_n))\]
	\item Si tengo una $f: \nat^n \to \nat$ que es constantemente
		$x_i + k$.
		\[f(x_1, \dots, x_n) = k(u^n_i(x_1, \dots, x_n))\]
\end{itemize}

Cómo sólo hice composiciones para construir la $f$ entonces esto debería bastar
para demostrar que las funciones constantes y las que suman constantes
pertenecen todas a $\Ccur_c$.

Para demostrar la ida necesito asegurarme que las funciones en $\Ccur_c$ sólo
pueden ser constantes (o sumar constantes). Si empezamos examinando los casos
base\dots

\begin{center}
	\begin{tabular}{lll}
		Primitiva  & Forma                  & $k$ \\ \toprule
		$n(x)=0$   & $f(x) = k$             & $0$ \\
		$s(x)=x+1$ & $f(x) = x + k = x + 1$ & $1$ \\
		$u^n_i(x_1, \dots, x_n)=x_i$ & $f(x_1, \dots, x_n) = x_i + k$
		& $0$
	\end{tabular}
\end{center}

\dots tenemos que todos ellos tienen o la forma $k$ o la de $x_i + k$. Ahora
examinemos nuestra regla de composición:

\begin{gather*}
	\text{Sea:} \\
	\hspace{2em} f: \nat^m \to \nat \\
	\hspace{2em} g_1, \dots, g_m: \nat^n \to \nat \\
	\text{Podemos construir $h: \nat^n \to \nat$ de la siguiente manera:} \\
	\hspace{2em} h(x_1, \dots, x_n) =
	f(g_1(x_1, \dots, x_n), \dots, g_m(x_1, \dots, g_n)) \\
\end{gather*}

Puedo tomar $f_1, \dots, f_n$ que tengan la forma $k$ o la forma $x_i + k$  y
analizar qué pasa si las compongo (paso inductivo).

\begin{center}
	\begin{tabular}{lcccc}
		& \multicolumn{2}{l}{$f_i(x_1, \dots, x_n) = k$}
		& \multicolumn{2}{r}{$f_i(x_1, \dots, x_n) = x_j + k$} \\
		\cmidrule{2-5}
		Composición
		& Forma & $k'$    & Forma     & $k'$ \\ \toprule
		$n(f_i(x_1, \dots, f_n))$
		& $k$   & $0$     & $k$       & $0$ \\
		$s(f_i(x_1, \dots, f_n))$
		& $k$   & $k + 1$ & $x_j + k$ & $k + 1$ \\
		$u^n_i(f_1(\dots), \ dots, f_i(\dots), \dots, f_n(\dots))$
		& $k$   & $k$     & $x_j + k$ & $k$ \\
		$f_i(f_1(\dots), \dots, f_n(\dots))$
		& $k$   & $k$     & $f_j(\dots) + k$ & $f_j(\dots) + k$
	\end{tabular}
\end{center}

Si bien el último caso puede parecer molesto, por la hipótesis inductiva
sabemos que $f_k$ tiene o bien forma $k$ o bien forma $x_i + k$, entonces
$k' = k_{f_j} + k_{f_i}$ y la forma se mantiene.

Entonces, si parto de funciones en $\Ccur_c$ y las compongo siempre voy a
llegar a una función de la forma $k$ o $x_i + k$. Y dado que las funciones
iniciales tienen también esa forma entonces todo $\Ccur_c$ la tiene también.

\subsubsection{Mostrar que existe una función primitiva recursiva que no está
en $\Ccur_c$}

$f_1(x, y)$ del ejercicio 2. Es primitiva recursiva porque está construida en
base a la recursión primitiva y a la composición de funciones. Pero depende de
ambos parámetros para emitir su resultado (a diferencia de todas las de
$\Ccur_c$).

\subsection{Mostrar que los predicados $\leq$, $\geq$, $=$, $\neq$, $<$ t $>$
$:\nat^2 \to \{0, 1\}$ están en cualquier clase \emph{PRC}}

Llamamos \emph{predicado} a cualquier función $p: \nat^n \to \{0, 1\}$,
escribimos $p(a_1, \dots, a_n)$ en lugar de $p(a_1, \dots, a_n)=1$ y decimos,
informalmente, en este caso, que ``$p(a_1, \dots, a_n)$ es verdadero''.

Vamos a empezar poniéndoles nombres a las funciones que queremos construir y
vamos a tener en cuenta las funciones que construímos en el ejercicio 2 para
facilitarnos la vida:

\begin{align*}
	f_1(x, y)=
	\begin{cases}
		1 & \text{si } x\leq y \\
		0 & \text{sino}
	\end{cases} &&
	f_2(x, y)=
	\begin{cases}
		1 & \text{si } x\geq y \\
		0 & \text{sino}
	\end{cases} &&
	f_3(x, y)=
	\begin{cases}
		1 & \text{si } x=y \\
		0 & \text{sino}
	\end{cases} \\ \\
	f_4(x, y)=
	\begin{cases}
		1 & \text{si } x\neq y \\
		0 & \text{sino}
	\end{cases} &&
	f_5(x, y)=
	\begin{cases}
		1 & \text{si } x<y \\
		0 & \text{sino}
	\end{cases} &&
	f_6(x, y)=
	\begin{cases}
		1 & \text{si } x>y \\
		0 & \text{sino}
	\end{cases}
\end{align*}

Y para hacernos la vida más fácil vamos a construir $\alpha(x)$ que es la
negación lógica en nuestro modelo.
\begin{align*}
	\alpha(0)     &= n(0) \\
	\alpha(t + 1) &= s(n(t))
\end{align*}

¡Y ahora sí!
\begin{align*}
	f_1(x, y) &= \alpha(x\dot-y) \\
	f_2(x, y) &= \alpha(y\dot-x) \\
	f_3(x, y) &= (x \leq y) \cdot (y \leq x) \\
	f_4(x, y) &= \alpha(x = y) \\
	f_5(x, y) &= \alpha(x \geq y)) \\
	f_6(x, y) &= \alpha(x \leq y))
\end{align*}

\emph{Nota:} $f_3$ puede parecer raro al no parecer una composición tan simple,
plantearlo prolijamente requiere una funcion auxiliar que cambia el orden de
los parámetros.

\subsection{Sean $\Ccur$ una clase \emph{PRC}, $f_1, \dots, f_k,
g:\nat^n\to\nat$ funciones en $\Ccur$ y $p_1, \dots, p_k:\nat^n\to\nat$
predicados disjuntos en $\Ccur$. Mostrar que la siguiente función también está
en $\Ccur$:}

\[
	h(x_1, \dots, x_n) =
	\begin{cases}
		f_1(x_1, \dots, x_n) & \text{si } p_1(x_1, \dots, x_n) \\
		\hspace{3em}\vdots \\
		f_k(x_1, \dots, x_n) & \text{si } p_k(x_1, \dots, x_n) \\
		g(x_1, \dots, x_n) & \text{sino} \\
	\end{cases}
\]

Observar que $h$ queda completamente determinada por este esquema.

\emph{Nota:} Al ser $p_1, \dots, p_k$ disjuntos no sucede $p_i(a_1, \dots, a_n)
= p_j(a_1, \dots, a_n) = 1$ con $i \neq j$ para ningún
$(a_1, \dots, a_n) \in \nat^n$.

Si podemos resolverlo para el caso de $k=1$ entonces podremos resolverlo para
cualquier $k$ arbitrario. Podemos construir $h(x_1, \dots, x_n)$ de la
siguiente forma:
\begin{align*}
	h(x_1, \dots, x_n)   &= h_1(x_1, \dots, x_n) \\
	h_i(x_1, \dots, x_n) &=
	\begin{cases}
		f_i(x_1, \dots, x_n) & \text{si } p_i(x_1, \dots, x_n) \\
		h_{i + 1}(x_1, \dots, x_n) & \text{sino}
	\end{cases} & \forall i \neq k+1 \\
	h_{k + 1}(x_1, \dots, x_n) &= g(x_1, \dots, x_n)
\end{align*}

Ahora sólo queda resolverlo para el caso de $k = 1$. ¡Cosa que ya hicimos en el
ejercicio 2 con $\max\{x, y\}$ y $\min\{x, y\}$! Repasemos esa solución:
construimos una función de decisión $d_i$ y la encapsulamos para construir el
$h_i$.
\begin{align*}
	d_i(x_1, \dots, x_n, 0)     &= f_i(x_1, \dots, x_n) \\
	d_i(x_1, \dots, x_n, t + 1) &= h_{i + 1}(x_1, \dots, x_n) \\
	\\
	h_i(x_1, \dots, x_n)        &= d_i(x_1, \dots, x_n, p(x_1, \dots, x_n))
	& \forall i \neq k+1 \\
	h_{k+1}(x_1, \dots, x_n)    &= g(x_1, \dots, x_n)
\end{align*}

Ahora tenemos todo listo, sólo queda enunciar a nuestro $h$:

\begin{align*}
	h(x_1, \dots, x_n) = h_1(x_1, \dots, x_n)
\end{align*}

\emph{Nota}: Una forma alternativa (y mucho más simple) que me dieron fué
construir \\ $h(x_1, \dots, x_n)$ de la siguiente manera:

\begin{alignat*}{3}
	h(x_1, \dots, x_n)
	=&& f_1(x_1, \dots, x_n) &\cdot p_1(x_1, \dots, x_n) \\
	+&& f_2(x_1, \dots, x_n) &\cdot p_2(x_1, \dots, x_n) \\
	&&&\vdotswithin{\cdot} \\
	+&& \ f_{k-1}(x_1, \dots, x_n) &\cdot p_{k-1}(x_1, \dots, x_n) \\
	+&& f_k(x_1, \dots, x_n) &\cdot p_k(x_1, \dots, x_n) \\
	+&& g(x_1, \dots, x_n)
	&\cdot \alpha(p_1(x_1, \dots, x_n) + \cdots + p_k(x_1, \dots, x_n))
\end{alignat*}

\end{document}
