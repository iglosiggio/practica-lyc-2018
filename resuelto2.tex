% !TEX TS-program = pdflatex
% !TEX encoding = UTF-8 Unicode

% This is a simple template for a LaTeX document using the "article" class.
% See "book", "report", "letter" for other types of document.

% align equations to the left
% use larger type; default would be 10pt
\documentclass[fleqn, 11pt]{article}

\usepackage[utf8]{inputenc} % set input encoding (not needed with XeLaTeX)

%%% Examples of Article customizations
% These packages are optional, depending whether you want the features they
% provide.
% See the LaTeX Companion or other references for full information.

%%% PAGE DIMENSIONS
\usepackage{geometry} % to change the page dimensions
\geometry{a4paper} % or letterpaper (US) or a5paper or....
% for example, change the margins to 2 inches all round
% \geometry{margin=2in}
% set up the page for landscape
% \geometry{landscape}
%   read geometry.pdf for detailed page layout information

% support the \includegraphics command and options
\usepackage{graphicx}

% Activate to begin paragraphs with an empty line rather than an indent
\usepackage[parfill]{parskip}

%%% PACKAGES
% for much better looking tables
\usepackage{booktabs}
% for better arrays (eg matrices) in maths
\usepackage{array}
% very flexible & customisable lists (eg. enumerate/itemize, etc.)
\usepackage{paralist}
% adds environment for commenting out blocks of text & for better verbatim
\usepackage{verbatim}
% make it possible to include more than one captioned figure/table in a single
% float
\usepackage{subfig}
% These packages are all incorporated in the memoir class to one degree or
% another...

%%% HEADERS & FOOTERS
\usepackage{fancyhdr} % This should be set AFTER setting up the page geometry
\pagestyle{fancy} % options: empty , plain , fancy
\renewcommand{\headrulewidth}{0pt} % customise the layout...
\lhead{}\chead{}\rhead{}
\lfoot{}\cfoot{\thepage}\rfoot{}

%%% SECTION TITLE APPEARANCE
\usepackage{sectsty}
% (See the fntguide.pdf for font help)
\allsectionsfont{\sffamily\mdseries\upshape}
% (This matches ConTeXt defaults)

%%% ToC (table of contents) APPEARANCE
% Put the bibliography in the ToC
\usepackage[nottoc, notlof, notlot]{tocbibind}
% Alter the style of the Table of Contents
\usepackage[titles, subfigure]{tocloft}
\renewcommand{\cftsecfont}{\rmfamily\mdseries\upshape}
\renewcommand{\cftsecpagefont}{\rmfamily\mdseries\upshape} % No bold!

\usepackage{calc}
\usepackage{lmodern}
\usepackage{amssymb}
\usepackage{amsmath}
\usepackage{mathdots}

\newcommand{\nat}{\mathbb{N}}
\newcommand{\Ccur}{\mathcal{C}}
\newcommand{\Scur}{\mathcal{S}}
\newcommand{\indef}{\downarrow}
\newcommand{\into}{\leftarrow}

\overfullrule=2em

%%% END Article customizations

%%% The "real" document content comes below...

\title{Resueltos Lógica y Computabilidad}
\author{Ignacio E. Losiggio}
% Activate to display a given date or no date (if empty), otherwise the current
% date is printed 
%\date{}

\begin{document}
\maketitle
\section{Práctica 2 --- Funciones $\Scur$--computables}

\subsection{Ejercicio 1}

\subsubsection{Definir \emph{macros} para las siguientes pseudo---instrucciones
(con su interpretación natural) e indicar en cada caso qué etiquetas se asumen
``frescas''}

\begin{itemize}
	\item $V_i \into k$
		\begin{align*}
			[R]\ & V_i \into V_i - 1 \\
			     & IF\ V_i \neq 0\ GOTO\ R \\
			     & V_i \into V_i + 1 \\
			     & \hspace{1.5em} \vdots
			       \hspace{2em}k\ \text{veces} \\
			     & V_i \into V_i + 1
		\end{align*}
		Se toma sólo la etiqueta $R$ cómo fresca.
	\item $V_i \into V_j + kl$
		\begin{align*}
			     & V_i \into k \\
			     & Z_a \into Z_a + 1 \\
			     & IF\ Z_a \neq 0\ GOTO\ C \\
			[S]\ & V_j \into V_j - 1 \\
			     & V_i \into V_i + 1 \\
			     & Z_a \into Z_a + 1 \\
			[C]\ & IF\ V_j \neq 0\ GOTO\ S \\
			     & IF\ Z_a \neq 0\ GOTO\ F \\
			[L]\ & V_j \into V_j + 1 \\
			[F]\ & Z_a \into Z_a - 1 \\
			     & IF\ Z_a \neq 0\ GOTO\ L
		\end{align*}
		Se toman las etiquetas $S$, $C$, $L$, $F$ y la variable
		$Z_a$ cómo frescas.
	\item $IF\ V_i = 0\ GOTO\ L$
		\begin{align*}
			     &IF\ V_i \neq 0\ GOTO\ C \\
			     &Z_a \into Z_a + 1 \\
			     &IF\ Z_a \neq 0\ GOTO\ L \\
			[C]\ &Z_a \into Z_a + 1 \\
		\end{align*}
		Se toman la etiqueta $C$ y la variable $Z_a$ cómo frescas.
	\item $GOTO\ L$
		\begin{align*}
			     &Z_a \into Z_a + 1 \\
			     &IF\ Z_a \neq 0\ GOTO\ L \\
		\end{align*}
		Se toma sólo la variable $Z_a$ cómo fresca.
\end{itemize}

\end{document}
