% !TEX TS-program = pdflatex
% !TEX encoding = UTF-8 Unicode

% This is a simple template for a LaTeX document using the "article" class.
% See "book", "report", "letter" for other types of document.

% align equations to the left
% use larger type; default would be 10pt
\documentclass[fleqn, 11pt]{article}

\usepackage[utf8]{inputenc} % set input encoding (not needed with XeLaTeX)

%%% Examples of Article customizations
% These packages are optional, depending whether you want the features they
% provide.
% See the LaTeX Companion or other references for full information.

%%% PAGE DIMENSIONS
\usepackage{geometry} % to change the page dimensions
\geometry{a4paper} % or letterpaper (US) or a5paper or....
% for example, change the margins to 2 inches all round
% \geometry{margin=2in}
% set up the page for landscape
% \geometry{landscape}
%   read geometry.pdf for detailed page layout information

% support the \includegraphics command and options
\usepackage{graphicx}

% Activate to begin paragraphs with an empty line rather than an indent
\usepackage[parfill]{parskip}

%%% PACKAGES
% for much better looking tables
\usepackage{booktabs}
% for better arrays (eg matrices) in maths
\usepackage{array}
% very flexible & customisable lists (eg. enumerate/itemize, etc.)
\usepackage{paralist}
% adds environment for commenting out blocks of text & for better verbatim
\usepackage{verbatim}
% make it possible to include more than one captioned figure/table in a single
% float
\usepackage{subfig}
% These packages are all incorporated in the memoir class to one degree or
% another...

%%% HEADERS & FOOTERS
\usepackage{fancyhdr} % This should be set AFTER setting up the page geometry
\pagestyle{fancy} % options: empty , plain , fancy
\renewcommand{\headrulewidth}{0pt} % customise the layout...
\lhead{}\chead{}\rhead{}
\lfoot{}\cfoot{\thepage}\rfoot{}

%%% SECTION TITLE APPEARANCE
\usepackage{sectsty}
% (See the fntguide.pdf for font help)
\allsectionsfont{\sffamily\mdseries\upshape}
% (This matches ConTeXt defaults)

%%% ToC (table of contents) APPEARANCE
% Put the bibliography in the ToC
\usepackage[nottoc, notlof, notlot]{tocbibind}
% Alter the style of the Table of Contents
\usepackage[titles, subfigure]{tocloft}
\renewcommand{\cftsecfont}{\rmfamily\mdseries\upshape}
\renewcommand{\cftsecpagefont}{\rmfamily\mdseries\upshape} % No bold!

\usepackage{calc}
\usepackage{lmodern}
\usepackage{amssymb}
\usepackage{amsmath}
\usepackage{mathdots}

\newcommand{\nat}{\mathbb{N}}
\newcommand{\Ccur}{\mathcal{C}}
\newcommand{\Scur}{\mathcal{S}}
\newcommand{\indef}{\uparrow}
\newcommand{\into}{\leftarrow}
\newcommand{\IF}{\text{IF}}
\newcommand{\GOTO}{\text{GOTO}}
\newcommand{\WHILE}{\text{WHILE}}
\newcommand{\THEN}{\text{THEN}}
\newcommand{\ELSE}{\text{ELSE}}

\overfullrule=2em

%%% END Article customizations

%%% The "real" document content comes below...

\title{Resueltos Lógica y Computabilidad}
\author{Ignacio E. Losiggio}
% Activate to display a given date or no date (if empty), otherwise the current
% date is printed 
%\date{}

\begin{document}
\maketitle
\section{Práctica 2 --- Funciones $\Scur$--computables}

\subsection{}

\subsubsection{Definir \emph{macros} para las siguientes pseudo-instrucciones
(con su interpretación natural) e indicar en cada caso qué etiquetas se asumen
``frescas''}

\begin{itemize}
	\item $V_i \into k$
		\begin{align*}
			[R]\ & V_i \into V_i - 1 \\
			     & \IF\ V_i \neq 0\ \GOTO\ R \\
			     & V_i \into V_i + 1 \\
			     & \hspace{1.5em} \vdots
			       \hspace{2em}k\ \text{veces} \\
			     & V_i \into V_i + 1
		\end{align*}
		Se toma sólo la etiqueta $R$ cómo fresca.
	\item $V_i \into V_j + k$
		\begin{align*}
			     & V_i \into k \\
			     & Z_a \into Z_a + 1 \\
			     & \IF\ Z_a \neq 0\ \GOTO\ C \\
			[S]\ & V_j \into V_j - 1 \\
			     & V_i \into V_i + 1 \\
			     & Z_a \into Z_a + 1 \\
			[C]\ & \IF\ V_j \neq 0\ \GOTO\ S \\
			     & \IF\ Z_a \neq 0\ \GOTO\ F \\
			[L]\ & V_j \into V_j + 1 \\
			[F]\ & Z_a \into Z_a - 1 \\
			     & \IF\ Z_a \neq 0\ \GOTO\ L
		\end{align*}
		Se toman las etiquetas $S$, $C$, $L$, $F$ y la variable
		$Z_a$ cómo frescas.
	\item $\IF\ V_i = 0\ \GOTO\ L$
		\begin{align*}
			     &\IF\ V_i \neq 0\ \GOTO\ C \\
			     &Z_a \into Z_a + 1 \\
			     &\IF\ Z_a \neq 0\ \GOTO\ L \\
			[C]\ &Z_a \into Z_a + 1 \\
		\end{align*}
		Se toman la etiqueta $C$ y la variable $Z_a$ cómo frescas.
	\item $\GOTO\ L$
		\begin{align*}
			     &Z_a \into Z_a + 1 \\
			     &\IF\ Z_a \neq 0\ \GOTO\ L \\
		\end{align*}
		Se toma sólo la variable $Z_a$ cómo fresca.
\end{itemize}

\subsubsection{Definir dos pseudo-programas distintos en el lenguaje $\Scur$
(usando las macros convenientes del punto anterior) que computen la función de
dos variables $f(x_1, x_y) = x_1 + x_2$. Par aalguno de los dos, expandir las
macros utilizadas prestando atención a la instanciación de variables y
etiquetas frescas.}

\begin{align*}
	\begin{aligned}
		     &Y   \into X_1 + 0 \\
		     &\GOTO\ B \\
		[A]\ &Y   \into Y   + 1 \\
		     &X_2 \into X_2 - 1 \\
		[B]\ &\IF\ X_2 \neq 0\ \GOTO\ A
	\end{aligned}
	&&
	\begin{aligned}\
		[A]\ &\IF\ X_1 = 0\ \GOTO\ B \\
		     &Y   \into Y   + 1 \\
		     &X_1 \into X_1 - 1 \\
		     &\GOTO\ A \\
		[B]\ &\IF\ X_2 = 0\ \GOTO\ E \\
		     &Y   \into Y   + 1 \\
		     &X_2 \into X_2 - 1 \\
		     &\GOTO\ B
	\end{aligned}
\end{align*}

Vamos a expandir la segunda de las formulaciones (por ser la que tiene macros
más simples).
\begin{align*}
	[A]\ &\IF\ X_1 \neq 0\ \GOTO\ Y \\
	     &Z_1 \into Z_1 + 1 \\
	     &\IF\ Z_1 \neq 0\ \GOTO\ B \\
	[Y]\ &Z_1 \into Z_1 + 1 \\
	     &Y   \into Y   + 1 \\
	     &X_1 \into X_1 - 1 \\
	     &Z_2 \into Z_2 + 1 \\
	     &\IF\ Z_2 \neq 0\ \GOTO\ A \\
	[B]\ &\IF\ X_2 \neq 0\ \GOTO\ Z \\
	     &Z_3 \into Z_3 + 1 \\
	     &\IF\ Z_3 \neq 0\ \GOTO\ E \\
	[Z]\ &Z_3 \into Z_3 + 1 \\
	     &Y   \into Y   + 1 \\
	     &X_2 \into X_2 - 1 \\
	     &Z_4 \into Z_4 + 1 \\
	     &\IF\ Z_4 \neq 0\ \GOTO\ B \\
\end{align*}

\subsubsection{Sea $P$ el programa en $\Scur$ que resulta de expandir todas las
macros en aguno de los códigos del punto anterior. Determinar cuál es la
función computada en cada caso:}

\begin{itemize}
	\item $\Psi^{(1)}_P : \nat   \to \nat$
		
		$f(x) = x$, se puede ver fácil desde planteo del ejercicio
		anterior, los parámetros no inicializados son ceros por lo que
		la función pedida se instancia como  $f(x_1, 0) = x + 0$ y se
		transforma nuestra suma en la función identidad.
	\item $\Psi^{(2)}_P : \nat^2 \to \nat$

		$f(x, y) = x + y$, la función pedida el ejercicio anterior.
	\item $\Psi^{(3)}_P : \nat^3 \to \nat$

		$f(x, y, z) = x + y$, dado que ignoramos el tercer parámetro en
		ambas formulaciones del programa.
\end{itemize}

\subsection{}

\subsubsection{Sea $\Ccur_\Scur = \{\Psi^{(n)}_P\ |\ P \text{\ es un programa
en } S, n \geq 1\} $ la clase de funciones $\Scur$-parciales computables.
Mostrar que $\Ccur_\Scur$ es una clase $PRC$}

Para mostrar que es $PRC$ necesito dar un programa para las iniciales y
demostrar que $\Ccur_\Scur$ está cerrado por composición y recursión primitiva.

Vamos primero por las iniciales:
\begin{align*}
	&n(x) = 0 && s(x) = x + 1 && u^n_i(x_1, \dots, x_n) = x_i \\
	&Z_1 \into Z_1 + 1
	&&Y \into X_1 + 1
	&&Y \into X_i + 0
\end{align*}

Y ahora veamos cómo resolver la composición, tomo $h : \nat^k \to \nat$ y
$g_1, \dots, g_k : \nat^n \to \nat$ que pertenezcan a $\Ccur_\Scur$ (y por lo
tanto tengan programas que podamos usar cómo macros):
\[
	f(x_1, \dots, x_n) = h(g_1(x_1, \dots, x_n), \dots,
	                       g_k(x_1, \dots, x_n))
\]
\begin{gather*}
	Z_1 \into g_1(X_1, \dots, X_n) \\
	\hspace{1.8em} \vdots \\
	Z_k \into g_k(X_1, \dots, X_n) \\
	Y   \into h(Z_1, \dots, Z_k)
\end{gather*}

Para que esto funcione tenemos que restringir a cada macro a dejar sus
variables de entrada intactas al finalizar su ejecución. Una forma de
mecanizarlo es que cada macro copie todas sus variables de entrada a variables
temporales ``frescas'' (que no se hayan usado ni se vayan a usar) y que cada
macro designe como variable de salida una variable temporal ``fresca''. La
única excepción a esto es el macro de asignación $V_i \into Expr$ que aunque el
resultado de $Expr$ esté en una variable ``fresca'' debe modificar $V_i$ para
cumplir con su tarea.

Dicho todo esto, vamos por la recursión primitiva, la cuál es más simple de lo
que parece, tomo $h : \nat^n \to \nat$ y $g : \nat^{n + 2} \to \nat$ que
pertenezcan a $\Ccur_\Scur$ (¡Osea que tenemos programas!):
\begin{align*}
	f(x_1, \dots, x_n, 0)     &= h(x_1, \dots, x_n) \\
	f(x_1, \dots, x_n, t + 1) &= g(f(x_1, \dots, x_n, t),
				       x_1, \dots, x_n, t)
\end{align*}
\begin{align*}
	     &Y \into h(X_1, \dots, X_n) \\
	[L]\ &\IF\ X_{n+1}\ = 0\ \GOTO\ E \\
	     &Z_1 \into Z_1 + 1 \\
	     &X_{n+1} \into X_{n+1} - 1 \\
	     &Y \into g(Y, X_1, \dots, X_n, Z_1) \\
	     &\GOTO\ L
\end{align*}

\subsubsection{Demostrar (sin definir un programa en $\Scur$) que la función
$* : \nat^2 \to \nat$ definida por $*(x, y) = x \cdot y$ es
$\Scur$-computable.}

En la práctica anterior la demostramos primitiva recursiva y sabemos que
$pr \subseteq PRC$ para cualquier conjunto $PRC$. En el punto anterior
demostramos que $\Ccur_\Scur$ era $PRC$, por lo que todas las primitivas
recursivas están allí, entre ellas $*$.

\subsubsection{Si $f : \nat^n \to \nat$ es una función primitiva recursiva ¿Qué
podemos decir acerca de la existencia de un programa en el lenguaje $\Scur$ que
la compute?}

Que existe (por las mismas razones del punto anterior).

\subsection{Decimos que un programa $P$ es \emph{autocontenido} si en cada
instrucción $\IF\ V \neq 0\ \GOTO\ L$ que ocurre en $P$, $L$ es una etiqueta
definida en $P$.}

\subsubsection{Demostrar que todo programa $P$ tiene un programa autocontenido
$P'$ equivalente ($P$ y $P'$ son programs equivalentes si
$\Psi^{(n)}_P = \Psi^{(n)}_{P'} \forall n \geq 1$).}

Supongamos $P$ un programa no autocontenido con una cantidad $k$ de etiquetas
``libres'' $L_1, \dots, L_k$ (etiquetas en uso pero sin lugar hacia dónde
saltar). Cómo el programa $P$ es finito usa una cantida finita de variables
temporales $Z_1, \dots, Z_n$ y tiene una cantidad finita de instrucciones
podemos construir $P'$ agregándole las siguientes instrucciones al final de
$P$:
\begin{align*}
	[L_1]\ &Z_{n+1} \into Z_{n+1} + 1 \\
	       & \hspace{2.7em} \vdots \\
	[L_k]\ &Z_{n+1} \into Z_{n+1} + 1
\end{align*}
Necesitamos sólo una variable ``fresca'' y cómo nunca modificamos a $Y$ el
resultado del programa $P$ no se modifica.

\subsubsection{Sean $P$ y $Q$ dos programas autocontenidos con etiquetas
disjuntas y sea $r : \nat^n \to \{0, 1\}$ un predicado primitivo recursivo.
Definir macros para las siguientes pseudo-instrucciones (con su interpretación
natural)}

\begin{itemize}
	\item $\IF\ r(V_1, \dots, V_n)\ \GOTO\ L$
		\begin{gather*}
			Z_a \into r(V_1, \dots, V_n) \\
			\IF\ Z_a \neq 0\ \GOTO\ L
		\end{gather*}
	\item $\IF\ r(V_1, \dots, V_n)\ \THEN\ P\ \ELSE\ Q$
		\begin{align*}
			     &\IF\ r(V_1, \dots, V_n)\ \GOTO\ P \\
			     &\text{; Acá van los contenidos de }Q \\
			     &\GOTO\ E \\
			[P]\ &\text{; Acá van los contenidos de }P
		\end{align*}
	\item $\WHILE\ r(V_1, \dots, V_n)\ P$
		\begin{align*}
			[L]\ &\IF\ \alpha(r(V_1, \dots, V_n))\ \GOTO\ E \\
			     &\text{; Acá van los contenidos de }P \\
			     &\GOTO\ L
		\end{align*}
\end{itemize}

\subsubsection{Dadas las funciones $f, g : \nat \to \nat$ definidas por:}

\begin{align*}
	f(x) =
	\begin{cases}
		1      & \text{si } x = 3 \\
		\indef & \text{en otro caso}
	\end{cases}  && g(x) = 2x
\end{align*}

Demostrar que es $\Scur$-parcial computable la función
\[
	h(x) = 
	\begin{cases}
		f(x) & \text{si } x \geq 5 \lor x = 3 \\
		g(x) & \text{en otro caso}
	\end{cases}
\]

Tengo que las operaciones $a = b$, $a \geq b$, $a \lor b$ y $a \cdot b$ son
p.r., por lo que sé que existen programas en $\Scur$ que las computan. También
tengo que la división por casos en una clase $PRC$ genera una función dentro de
la misma clase. Dado todo esto ya puedo asegurar que $g(x)$ es $\Scur$-parcial
computable y que si $f(x)$ lo fuera entonces $h(x)$ también.

Me queda demostrar que en $\Scur$ los programas (a veces) se cuelgan. Defino
$j(x) = \indef \forall x$ y reescribo $f(x)$ cómo una composición con $j(x)$.
\[
	f(x) =
	\begin{cases}
		1    & \text{si } x = 3 \\
		j(x) & \text{en otro caso}
	\end{cases}
\]

Y sólo me queda ofrecer el código de $j(x)$:
\[[A]\ \GOTO\ A\]

\subsection{}

\subsubsection{Se definen las siguientes variantes del lenguaje $\Scur$:}

\begin{itemize}
	\item $\Scur_1$: Igual que $\Scur$ pero sin la instrucción
	$V \into V + 1$
	\item $\Scur_2$: Igual que $\Scur$ pero sin la instrucción
	$\IF\ V \neq 0\ \GOTO\ L$
	\item $\Scur_3$: Igual que $\Scur$ pero sin la instrucción
	$V \into V \dot- 1$
\end{itemize}

Demostrar que para cada uno de estos lenguajes existe al menos una función
$\Scur$-parcial computable que no es computable en este nuevo lenguaje.

Para el primer caso podemos tomar $f(x) = 1$. Como $Y$ arranca valiendo $0$ y
sólo podemos decrementarlo entonces no es posible hacer que le valor de $Y$
supere el $0$ con las instrucciones que nos quedan.

Para el segundo caso podemos tomar la función $f(x) = x$. Para hacer que el
valor de $Y$ llegue a $x$ hay que ejecutar \emph{al menos} $x$ instrucciones.
Dado que no tenemos ninguna forma de controlar el flujo del programa es obvio
que todo programa en $\Scur_2$ siempre va a ejecutar la misma cantidad de
instrucciones. Entonces finalmente podemos decir que para cualquier programa
$P$ en $\Scur_2$ existe un $x$ mayor a su longitud (la cantida de de
instrucciones). Por lo que no se puede construir un programa que sea $f(x) = x$
para cualquier $x$. Otra respuesta posible (y más simple) era hablar de la
función qué se indefine siempre. Cómo en $\Scur_2$ todos los programas tienen
una cantida finita de instrucciones entonces todos son totales, el programa que
se indefine para toda entrada no es construible.

Para $\Scur_3$ notemos primero que al iniciar el programa todo el estado no
inicializado ($X_n, X_{n+1}, \dots Y, Z_1, Z_2, Z_3, \dots$) toma el valor 0.
Esto hace que todas esas variables no generen saltos en las instrucciones
$\IF\ V \neq 0\ \GOTO\ L$. Podemos incrementarlas en $1$ y allí sólo van a
producir saltos en esas instrucciones, pero una vez hecho esto no pueden
\emph{dejar de hacerlo}. Por ende el flujo que toma el programa depende
únicamente que si las variables de entrada son $0$ o son $> 0$. No existe forma
de que una $V \neq 0$ en un tiempo $t$ pase a ser igual a $0$ en un tiempo
$t + k$. Es decir, los programas en $\Scur_3$ dependen de cuáles de la
finita cantidad de entradas que leen son $\neq 0$ pero no de los valores que
ellas tienen. Es por esto que no puede construirse $f(x) = x$.

\subsubsection{Sea $\Scur'$ el lenguaje de programación definido como $\Scur$
salvo que sus instrucciones (etiquetadas o no) son de los siguientes tipos (con
su interpretación antural):}
\begin{gather*}
	V \into V' \\
	V \into V + 1 \\
	\IF\ V \neq V'\ \GOTO\ L
\end{gather*}
Demostrar que una función es parcial computable en $\Scur'$ si y solo si lo es
en $\Scur$.

Cómo los lenguajes estos están cerrados por las instrucciones básicas y la
secuencia alcanzaría con ofrecer macros que representen a las instrucciones
básicas de c/u en el otro para mostrar su equivalencia.

\begin{tabular}{l l}
	$\Scur'$                    & Macro en $\Scur$ \\ \toprule
	$V \into V'$                & $V \into V' + 0$ \\
	$V \into V + 1$             & $V \into V  + 1$ \\
	$\IF\ V \neq V'\ \GOTO\ L$  & $Z_a \into V \dot- V' + V' \dot- V$ \\
	                            & $\IF\ Z_a \neq 0\ \GOTO\ L$ \\
\end{tabular}

\begin{tabular}{l l}
	$\Scur$                     & Macro en $\Scur'$ \\ \toprule
	$V \into V + 1$             & $V \into V + 1$ \\
	$\IF\ V \neq 0\ \GOTO\ L$ & $\IF\ V \neq Z_a\ \GOTO\ L$ \\
	$V \into V \dot- 1$         &
	$\begin{aligned}
		     &\IF Z_a \neq V\ \GOTO\ A \\
		     &Z_a \into Z_a + 1 \\
		     &\IF\ Z_a \neq V\ \GOTO\ E \\
		[A]\ &Z_a \into Z_a + 1 \\
		     &\IF\ Z_a \neq V\ \GOTO\ B \\
		     &Z_a \into Z_a + 1 \\
		     &\IF\ Z_a \neq V\ \GOTO\ E \\
		[B]\ &Z_a \into Z_a + 1 \\
		     &Z_b \into Z_b + 1 \\
		     &\IF\ Z_a \neq V\ GOTO\ B \\
		[E]\ &V \into Z_b
	\end{aligned}$
\end{tabular}

\subsection{}

\subsubsection{Demostrar que si $p : \nat^{n+1} \to \{0, 1\}$ es un predicado
$\Scur$-computable (total), entonces es $\Scur$-parcial computable:}
\[
	\text{mínimoNA}_p(x_1, \dots, x_n, y) =
	\begin{cases}
		\min \{t | y \leq t \land p(x_1, \dots, x_n, t)\}
			& \text{si existe algún } t \\
		\indef & \text{en otro caso}
	\end{cases}
\]

La forma más simple de demostrarlo es dar un programa:
\begin{align*}
	     &Y \into y \\
	[A]\ &\IF\ p(X_1, \dots, X_n, Y) \neq 0\ \GOTO\ E \\
	     &Y \into Y + 1 \\
	     &\GOTO\ A
\end{align*}

\subsubsection{Mostrar, usando el resultado anterior, que si
$f : \nat \to \nat$ es biyectiva y $\Scur$-computable (total) entonces también
lo es su inversa $f^{-1}$.}

Partiendo del ejercicio anterior y con unos reemplazos simples obtenemos algo
similar a esto:
\begin{gather*}
	p(x, y) = (f(x) = y) \\
	f^{-1}(x) = \text{mínimoNA}_p(x, 0)
\end{gather*}
Cómo sabemos que $f$ es total entonces siempre existe ese $t$ y $f^{-1}$
resulta también total.

\subsection{Un programa $P$ en el lenguaje $\Scur$ con instrucciones $I_1, I_2,
\dots, I_n$ se dice \emph{optimista} si $\forall i = 1, \dots, n$, si $I_i$ es
la instrucción $\IF\ V \neq 0\ \GOTO\ L$ entonces $L$ no aparece como etiqueta
de ninguna instrucción $I_j$ con $j \leq i$.}

Demostrar que el siguiente predicado es primitivo recursivo:
\[
	r(x) =
	\begin{cases}
		1 & \text{si el programa cuyo número es $x$ es optimista} \\
		0 & \text{caso contrario}
	\end{cases}
\]



\end{document}
